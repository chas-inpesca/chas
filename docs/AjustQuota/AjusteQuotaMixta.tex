\documentclass[]{article}
\usepackage[T1]{fontenc}
\usepackage{lmodern}
\usepackage{amssymb,amsmath}
\usepackage{ifxetex,ifluatex}
\usepackage{fixltx2e} % provides \textsubscript
% use upquote if available, for straight quotes in verbatim environments
\IfFileExists{upquote.sty}{\usepackage{upquote}}{}
\ifnum 0\ifxetex 1\fi\ifluatex 1\fi=0 % if pdftex
  \usepackage[utf8]{inputenc}
\else % if luatex or xelatex
  \ifxetex
    \usepackage{mathspec}
    \usepackage{xltxtra,xunicode}
  \else
    \usepackage{fontspec}
  \fi
  \defaultfontfeatures{Mapping=tex-text,Scale=MatchLowercase}
  \newcommand{\euro}{€}
\fi
% use microtype if available
\IfFileExists{microtype.sty}{\usepackage{microtype}}{}
\usepackage[margin=1in]{geometry}
\usepackage{color}
\usepackage{fancyvrb}
\newcommand{\VerbBar}{|}
\newcommand{\VERB}{\Verb[commandchars=\\\{\}]}
\DefineVerbatimEnvironment{Highlighting}{Verbatim}{commandchars=\\\{\}}
% Add ',fontsize=\small' for more characters per line
\usepackage{framed}
\definecolor{shadecolor}{RGB}{248,248,248}
\newenvironment{Shaded}{\begin{snugshade}}{\end{snugshade}}
\newcommand{\KeywordTok}[1]{\textcolor[rgb]{0.13,0.29,0.53}{\textbf{{#1}}}}
\newcommand{\DataTypeTok}[1]{\textcolor[rgb]{0.13,0.29,0.53}{{#1}}}
\newcommand{\DecValTok}[1]{\textcolor[rgb]{0.00,0.00,0.81}{{#1}}}
\newcommand{\BaseNTok}[1]{\textcolor[rgb]{0.00,0.00,0.81}{{#1}}}
\newcommand{\FloatTok}[1]{\textcolor[rgb]{0.00,0.00,0.81}{{#1}}}
\newcommand{\CharTok}[1]{\textcolor[rgb]{0.31,0.60,0.02}{{#1}}}
\newcommand{\StringTok}[1]{\textcolor[rgb]{0.31,0.60,0.02}{{#1}}}
\newcommand{\CommentTok}[1]{\textcolor[rgb]{0.56,0.35,0.01}{\textit{{#1}}}}
\newcommand{\OtherTok}[1]{\textcolor[rgb]{0.56,0.35,0.01}{{#1}}}
\newcommand{\AlertTok}[1]{\textcolor[rgb]{0.94,0.16,0.16}{{#1}}}
\newcommand{\FunctionTok}[1]{\textcolor[rgb]{0.00,0.00,0.00}{{#1}}}
\newcommand{\RegionMarkerTok}[1]{{#1}}
\newcommand{\ErrorTok}[1]{\textbf{{#1}}}
\newcommand{\NormalTok}[1]{{#1}}
\usepackage{graphicx}
% Redefine \includegraphics so that, unless explicit options are
% given, the image width will not exceed the width of the page.
% Images get their normal width if they fit onto the page, but
% are scaled down if they would overflow the margins.
\makeatletter
\def\ScaleIfNeeded{%
  \ifdim\Gin@nat@width>\linewidth
    \linewidth
  \else
    \Gin@nat@width
  \fi
}
\makeatother
\let\Oldincludegraphics\includegraphics
{%
 \catcode`\@=11\relax%
 \gdef\includegraphics{\@ifnextchar[{\Oldincludegraphics}{\Oldincludegraphics[width=\ScaleIfNeeded]}}%
}%
\ifxetex
  \usepackage[setpagesize=false, % page size defined by xetex
              unicode=false, % unicode breaks when used with xetex
              xetex]{hyperref}
\else
  \usepackage[unicode=true]{hyperref}
\fi
\hypersetup{breaklinks=true,
            bookmarks=true,
            pdfauthor={Luis Cubillos},
            pdftitle={Ajuste de las cuotas de sardina común y anchoveta considerando la composición de las capturas},
            colorlinks=true,
            citecolor=blue,
            urlcolor=blue,
            linkcolor=magenta,
            pdfborder={0 0 0}}
\urlstyle{same}  % don't use monospace font for urls
\setlength{\parindent}{0pt}
\setlength{\parskip}{6pt plus 2pt minus 1pt}
\setlength{\emergencystretch}{3em}  % prevent overfull lines
\setcounter{secnumdepth}{0}

\title{Ajuste de las cuotas de sardina común y anchoveta considerando la
composición de las capturas}
\author{Luis Cubillos}
\date{10 de diciembre de 2014}

\begin{document}

\begin{center}
\huge Ajuste de las cuotas de sardina común y anchoveta considerando la
composición de las capturas \\[0.2cm]
\end{center}
\begin{center}
\large \emph{Luis Cubillos}\\[0.1cm]
\end{center}
\begin{center}
\large \emph{10 de diciembre de 2014} \\
\end{center}
\normalsize


\section{Introduction}\label{introduction}

En la actualidad, el manejo de pesquerías mediante el control de cuotas
de captura de especies separadas es ampliamente aceptado e
institucionalizado. Aunque el manejo mediante cuotas puede ser efectivo
en determinadas pesquerías, cuando se trata de pesquerías
multiespecíficas mixtas (PMM) podría ser poco efectivas. En efecto, en
PPM normalmente se acepta que se siga capturando hasta que todas las
cuotas hayan sido cubiertas. Asimismo, es común que algunas cuotas se
superen debido a que aún hay especies por capturas, o bien genera
descarte y malas prácticas.

En PPM, el balance óptimo de la contribución relativa de las especies a
la captura total se debe a multiples factores que están relacionados con
las operaciones de pesca y por el grado de mezcla que se de en
determinadas zonas de pesca. Este es el caso de la pesquería de sardina
común y anchoveta en la zona centro-sur de Chile. Ambos recursos son
especies objetivo de la flota artesanal e industrial, son recursos de
tamaño similar que forman agregaciones y cardúmenes mezclados e
imposibles de separar acústicamente (Gerlotto et al., 2004), y tiene
igual precio.

En este trabajo se propone un sistema de ajuste de cuotas sobre la base
de la contribución relativa de cada especie al desembarque que fue
registrada en el pasado reciente.

\section{Metodología}\label{metodologia}

La metodología se basa en un sistema de dos recursos que conforman una
pesquería mixta, y para los cuales se establecen cuotas de captura
biológicamente aceptables (CBA) por separado. Si se denomina por $Qs$ y
$Qa$ a la CBA del recurso $s$ y $a$, respectivamente; entonces resta por
conocer la composición de la captura total de la flota y que corresponde
a una fracción $p \ [0,1]$ para uno de los recursos. De esta manera, se
tendrá que el aporte del recurso $s$ a la captura total será $p$ y la
del otro recurso será su complemento $1-p$. Si $p=0.55$ entonces la
razón entre los recursos será $a/s= 0.45/0.55$.

La composición de especies puede variar por múltiples factores entre un
nivel inferior $l$ y otro superior $u$. Por ejemplo si $l=0.5$ y
$u=0.6$, entonces la composición específica entre $s$ y $a$ podría
fluctuar entre $a/s=0.5/0.5$ y $a/s=0.5/0.6$. Para la pesquería se asume
que los límites $l$ y $u$ son conocidos, y que la composición de la
captura viene dada por $p=\in [l,u]$. Con el objeto de proceder con el
ajuste de las cuotas, el procedimiento asume que bajo el sistema de
cuota la pesquería se detendrá cuando se logre completar una de las
cuotas. En otras palabras $Qs$, $Qa$, $l$, y $u$ son conocidos, mientras
que la composición final $p$ puede ser determinada en el rango $[l,u]$.

Al considerar que las capturas efectivas de los recursos pueden ser
menor o igual que las respectivas cuotas, i.e.,

\[Ca \leq Qa\] y

\[Cs \leq Qs\]

La captura total que conforma la mezcla es $Ca+Cs$, y por definición la
composición de la captura es $p$ asociada a $Cs$ es:

\[\frac{Cs}{Ca+Cs}=p\]

lo que equivale a:

\[pCa-(1-p)Cs=0\]

Esta última ecuación es una línea recta que pasa por los puntos
($Ca,Cs$) con una pendiente $b=p/(1-p)$ e intercepto $a=0$. Ya que $p$
varía en el intervalo $[l,u]$, la pendiente podría variar en el
intervalo:

\[[\frac{l}{1-l},\frac{u}{1-u}]\]

Las restricciones de este sistema de ecuaciones se muestran en la Figura
1, suponiendo que $Qs=11$, $Qa=25$, $l=0.3$ y $u=0.6$. Si el límite
superior es $u=0.6$ y al considerar que es más factible que la cuota de
$Qs$ se complete primero, entonces el desembarque final de $Qa$ debería
ser ajustado a que inevitablemente será $Ca=16,5$. Sin embargo, si se
considera $l=0.3$, entonces el ajuste debería ser más drástico
($Ca=4,7$).En consecuencia, la cuota mixta ($Qm=Cs+Ca$) podría fluctuar
entre 15,7 y 27,5, en vez de la suma inicial de cuotas de 36.

\includegraphics{./AjusteQuotaMixta_files/figure-latex/unnamed-chunk-2.pdf}

Al valorar la contribución de cada especie, y al suponer que la especie
con menor captura tiene mayor valor entonces se podría plantear que el
beneficio de la captura conjunta será:

\[Im=Ps*Cs+Pa*Ca\]

Al suponer que el valor de la conservación viene dado por el inverso de
la cuota, entonces $Ps=1/Qs$ y $Pa=1/Qa$. Esto implica que si se logra
extraer la totalidad de la cuota de cada especie, entonces el máximo
beneficio conjunto será igual a 2 ($Im=2$). Otra forma de valorar es por
el inverso de la biomasa con el objeto de asignar mayor peso a la
especie con menor biomasa. Sin embargo, ya que en las pesquerías de
peces pelágicos las cuotas son proporcional a la biomasa, es preferible
utilizar el inverso de la cuota.

Las restricciones del sistema de ecuaciones son:

\[Im=PsCs+PaCa=Max!\] \[Ca \leq Qa\] \[Cs \leq Qs\] \[pCa-(1-p)Cs=0\]

La optimización para encontrar el valor de la composición de la captura
considera los siguientes casos:

\begin{itemize}
\itemsep1pt\parskip0pt\parsep0pt
\item
  Caso I : $Qa/Qs < l/(1-l)$
\item
  Caso II : $l/(1-l) \leq Qa/Qs \leq u/(1-u)$
\item
  Caso III: $u/(1-u) < Qa/Qs$
\end{itemize}

En estos casos, la solución óptima para la composición de la captura
($p^{o}$) es:

\begin{itemize}
\itemsep1pt\parskip0pt\parsep0pt
\item
  Caso I : $p^{o}=l$
\item
  Caso II : $p^{o}=Qa/(Qa+Qs)$
\item
  CAso III: $p^{o}=u$
\end{itemize}

Las capturas ópimas serían:

\begin{itemize}
\itemsep1pt\parskip0pt\parsep0pt
\item
  Caso I : $[Ca^{o},Cs^{o}]=[((1-l)/l)Qa,Qa]$
\item
  Caso II : $[Ca^{o},Cs^{o}]=[Qs,Qa]$
\item
  Caso III: $[Ca^{o},Cs^{o}]=[Qs,(u/(1-u))Qs]$
\end{itemize}

El beneficio de las capturas totales sería:

\begin{itemize}
\itemsep1pt\parskip0pt\parsep0pt
\item
  Caso I : $Im^{o}=(Ps((1-l)/l)+Pa)Qa$
\item
  Caso II : $Im^{o}=PsQs+PaQa$
\item
  Caso III: $Im^{o}=(Ps+Pa(u/(1-u)))Qs$
\end{itemize}

A modo de ejemplo, si se considera $Qa=42$, $Qs=572$, $Pa=0,0238$,
$Ps=0,0017$, $l=0,08$ y $u=0,38$, entonces se tiene como resultado que
la composición de la captura óptima es $p^{o}=0,08$, luego $Ca=42$,
$Cs=483$, $Qm=Ca+Cs=525$, y máximo beneficio de la conservación
$Im=1,84$ muy próximo a 2.

\section{La composición de captura
históricas}\label{la-composicion-de-captura-historicas}

La pesquería de anchoveta y sardina común es un caso de pesquería mixta
multiespecifica, y la mezcla de especie puede ser debida tanto a
interacción tecnológica como ecológica. La interacción ecológica es más
dificil de demostrar, por ejemplo competencia. Sin embargo es más fácil
considerar que la mezcla de especies se debe a la interacción
tecnológica, ya que para el pescador no hay preferencias por una u otra
especie ya que el destino de la captura es para la reducción y tienen el
mismo precio. La preferencia, sin embargo, es un tema complejo ya que
cuando hay restricciones por agotamiento de la cuota de una de las
especies, el pescador preferirá capturar la especie que tiene un saldo
de la cuota.

Se utilizó los datos de capturas mensuales del periodo 1998-2013 para
calcular la razón entre la captura de anchoveta y la captura de sardina
($Ca/Cs$) mediante un modelo de efectos mixtos y considerando el año
como factor aleatoriocon intercepto cero.

Los resultados muestran que los límites de confianza de 95\% para la
composición de especies fluctúa entre 0,44 y 0,96 con un estimado de 0,7
para los efectos fijos, mientras que para el efecto aleatorio asociado a
los años el límite de confianza de 95\% fluctúa entre 0,36 y 0,76 con un
estimado de 0,52.

\begin{Shaded}
\begin{Highlighting}[]
\KeywordTok{intervals}\NormalTok{(m4,}\DataTypeTok{level=}\FloatTok{0.95}\NormalTok{)}
\end{Highlighting}
\end{Shaded}

\begin{verbatim}
## Approximate 95% confidence intervals
## 
##  Fixed effects:
##              lower    est.     upper
## (Intercept) 233.48 843.608 1453.7337
## sardina       0.44   0.699    0.9579
## attr(,"label")
## [1] "Fixed effects:"
## 
##  Random Effects:
##   Level: as.factor(YY) 
##              lower   est. upper
## sd(sardina) 0.3618 0.5227 0.755
## 
##  Within-group standard error:
## lower  est. upper 
##  9149  9554  9976
\end{verbatim}

Los cambios anuales han fluctuado entre 0,078 en el 2012 y 1,848 en el
2007 (Figura). En los años más recientes (2008-2013), la razón de
capturas fluctúa entre 0,078 y 0,378. Estos valores pueden ser
considerado como límite inferior (l=0,078) y superior (u=0,378) para
ajustar las cuotas de captura.

\includegraphics{./AjusteQuotaMixta_files/figure-latex/unnamed-chunk-5.pdf}

\includegraphics{./AjusteQuotaMixta_files/figure-latex/unnamed-chunk-6.pdf}

\begin{Shaded}
\begin{Highlighting}[]
\NormalTok{tb1}
\end{Highlighting}
\end{Shaded}

\begin{verbatim}
##       Year  rate2
##  [1,] 1998 0.8100
##  [2,] 1999 1.0380
##  [3,] 2000 0.5790
##  [4,] 2001 0.4055
##  [5,] 2002 0.6179
##  [6,] 2003 1.0640
##  [7,] 2004 1.0415
##  [8,] 2005 1.6448
##  [9,] 2006 0.6740
## [10,] 2007 1.8480
## [11,] 2008 0.3460
## [12,] 2009 0.3220
## [13,] 2010 0.3770
## [14,] 2011 0.1250
## [15,] 2012 0.0780
## [16,] 2013 0.2120
\end{verbatim}

\section{Discusión}\label{discusion}

\section{Anexo}\label{anexo}

El algoritmo de optimización es el siguiente:

\begin{Shaded}
\begin{Highlighting}[]
\NormalTok{comp <-}\StringTok{ }\NormalTok{function()\{}
  \NormalTok{if(Qr<}\StringTok{ }\NormalTok{A)\{}
        \NormalTok{p=l}
        \NormalTok{CaCs =}\StringTok{ }\KeywordTok{c}\NormalTok{(((}\DecValTok{1}\NormalTok{-l)/l)*Qa,Qa)}
        \NormalTok{Im =}\StringTok{ }\NormalTok{(Ps*((}\DecValTok{1}\NormalTok{-l)/l)+Pa)*Qa}
        \NormalTok{\}}
    \NormalTok{if(Qr>B)\{}
        \NormalTok{p=u}
        \NormalTok{CaCs =}\StringTok{ }\KeywordTok{c}\NormalTok{(Qs,(u/(}\DecValTok{1}\NormalTok{-u))*Qs)}
        \NormalTok{Im =}\StringTok{ }\NormalTok{(Ps+Pa*(u/(}\DecValTok{1}\NormalTok{-u)))*Qs}
        \NormalTok{\}}
     \NormalTok{if(Qr>=A &&}\StringTok{ }\NormalTok{Qr<=B)\{}
        \NormalTok{p=Qa/(Qs+Qa)}
        \NormalTok{CaCs =}\StringTok{ }\KeywordTok{c}\NormalTok{(Qs,Qa)}
        \NormalTok{Im =}\StringTok{ }\NormalTok{Ps*Qs+Pa*Qa}
        \NormalTok{\}}
    \NormalTok{out <-}\StringTok{ }\KeywordTok{list}\NormalTok{()}
    \NormalTok{out$p =}\StringTok{ }\NormalTok{p}
    \NormalTok{out$CaCs=CaCs}
    \NormalTok{out$Im=Im}
    \KeywordTok{return}\NormalTok{(out)}
\NormalTok{\}}
\end{Highlighting}
\end{Shaded}

Un ejemplo de aplicación es el siguiente

\begin{Shaded}
\begin{Highlighting}[]
\NormalTok{Qa =}\StringTok{ }\DecValTok{42}  \CommentTok{#Cuota de la especie a}
\NormalTok{Qs =}\StringTok{ }\DecValTok{572}  \CommentTok{#Cuota de la especie b}
\NormalTok{Pa =}\StringTok{ }\DecValTok{1}\NormalTok{/Qa  }\CommentTok{#Valor de la especie a por conservación}
\NormalTok{Ps =}\StringTok{ }\DecValTok{1}\NormalTok{/Qs  }\CommentTok{#Valor de la especie b por conservación}
\NormalTok{l=}\FloatTok{0.08}     \CommentTok{#limite inferior razon anchoveta/sardina}
\NormalTok{u=}\FloatTok{0.38}     \CommentTok{#Límite superior razón anchoveta/sardina}
\CommentTok{#Balance y ajuste de las cuotas separadas}
\NormalTok{Qr =}\StringTok{ }\NormalTok{Qa/Qs}
\NormalTok{A =}\StringTok{ }\NormalTok{(l/(}\DecValTok{1}\NormalTok{-l))}
\NormalTok{B =}\StringTok{ }\NormalTok{(u/(}\DecValTok{1}\NormalTok{-u))}
\NormalTok{sa <-}\StringTok{ }\KeywordTok{comp}\NormalTok{()}
\NormalTok{Qmixta <-}\StringTok{ }\NormalTok{sa$CaCs[}\DecValTok{1}\NormalTok{]+sa$CaCs[}\DecValTok{2}\NormalTok{]}
\NormalTok{Cs<-sa$CaCs[}\DecValTok{1}\NormalTok{]}
\NormalTok{Ca<-sa$CaCs[}\DecValTok{2}\NormalTok{]}
\NormalTok{Qmixta}
\end{Highlighting}
\end{Shaded}

\begin{verbatim}
## [1] 525
\end{verbatim}

\end{document}
