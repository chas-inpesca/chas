\documentclass[18pt,]{article}
\usepackage{lmodern}
\usepackage{amssymb,amsmath}
\usepackage{ifxetex,ifluatex}
\usepackage{fixltx2e} % provides \textsubscript
\ifnum 0\ifxetex 1\fi\ifluatex 1\fi=0 % if pdftex
  \usepackage[T1]{fontenc}
  \usepackage[utf8]{inputenc}
\else % if luatex or xelatex
  \ifxetex
    \usepackage{mathspec}
  \else
    \usepackage{fontspec}
  \fi
  \defaultfontfeatures{Ligatures=TeX,Scale=MatchLowercase}
    \setmainfont[]{Arial}
    \setmonofont[Mapping=tex-ansi]{Arial}
\fi
% use upquote if available, for straight quotes in verbatim environments
\IfFileExists{upquote.sty}{\usepackage{upquote}}{}
% use microtype if available
\IfFileExists{microtype.sty}{%
\usepackage{microtype}
\UseMicrotypeSet[protrusion]{basicmath} % disable protrusion for tt fonts
}{}
\usepackage[margin=1in]{geometry}
\usepackage{hyperref}
\hypersetup{unicode=true,
            pdftitle={Ecuaciones modelo estimacion sardina y anchoveta MSE Inpesca},
            pdfauthor={Artega y asociados},
            pdfborder={0 0 0},
            breaklinks=true}
\urlstyle{same}  % don't use monospace font for urls
\usepackage{graphicx,grffile}
\makeatletter
\def\maxwidth{\ifdim\Gin@nat@width>\linewidth\linewidth\else\Gin@nat@width\fi}
\def\maxheight{\ifdim\Gin@nat@height>\textheight\textheight\else\Gin@nat@height\fi}
\makeatother
% Scale images if necessary, so that they will not overflow the page
% margins by default, and it is still possible to overwrite the defaults
% using explicit options in \includegraphics[width, height, ...]{}
\setkeys{Gin}{width=\maxwidth,height=\maxheight,keepaspectratio}
\IfFileExists{parskip.sty}{%
\usepackage{parskip}
}{% else
\setlength{\parindent}{0pt}
\setlength{\parskip}{6pt plus 2pt minus 1pt}
}
\setlength{\emergencystretch}{3em}  % prevent overfull lines
\providecommand{\tightlist}{%
  \setlength{\itemsep}{0pt}\setlength{\parskip}{0pt}}
\setcounter{secnumdepth}{0}
% Redefines (sub)paragraphs to behave more like sections
\ifx\paragraph\undefined\else
\let\oldparagraph\paragraph
\renewcommand{\paragraph}[1]{\oldparagraph{#1}\mbox{}}
\fi
\ifx\subparagraph\undefined\else
\let\oldsubparagraph\subparagraph
\renewcommand{\subparagraph}[1]{\oldsubparagraph{#1}\mbox{}}
\fi

%%% Use protect on footnotes to avoid problems with footnotes in titles
\let\rmarkdownfootnote\footnote%
\def\footnote{\protect\rmarkdownfootnote}

%%% Change title format to be more compact
\usepackage{titling}

% Create subtitle command for use in maketitle
\newcommand{\subtitle}[1]{
  \posttitle{
    \begin{center}\large#1\end{center}
    }
}

\setlength{\droptitle}{-2em}
  \title{Ecuaciones modelo estimacion sardina y anchoveta MSE Inpesca}
  \pretitle{\vspace{\droptitle}\centering\huge}
  \posttitle{\par}
  \author{Artega y asociados}
  \preauthor{\centering\large\emph}
  \postauthor{\par}
  \predate{\centering\large\emph}
  \postdate{\par}
  \date{Septiembre 2017}

\usepackage{placeins}
\usepackage[utf8]{inputenc}
\renewcommand{\contentsname}{Indice documento}
\renewcommand{\listfigurename}{Indice de figuras}
\renewcommand{\listtablename}{Indice de tablas}

\begin{document}
\maketitle

\section{Indicadores poblacionales por
especie}\label{indicadores-poblacionales-por-especie}

\emph{Biomasa total} \[
BT_{i,j}^{sard}=w_{i,j}N_{i,j};\hspace{2cm}   BT_{i,j}^{anch}=w_{i,j}N_{i,j}
\]

\emph{Biomasa adulta} \[
BA_{i,j}^{sard}=w_{i,j}\mu_{i,j}N_{i,j}\hspace{2cm} BA_{i,j}^{anch}=w_{i,j}\mu_{i,j}N_{i,j}
\] Donde \(w_{i,j}\) corresponde a peso y a la madurez \(\mu_{i,a}\)
correspondiente al año i y edad j.

\emph{Biomasa desovante sardina} \[
BD_{i,j}^{sard}=w_{i,j}\mu_{i,j}N_{i,j}\exp^{(-(T_{s})Z_{i,j})} \hspace{2cm}BD_{i,j}^{anch}=w_{i,j}\mu_{i,j}N_{i,j}\exp^{(-(T_{s})Z_{i,j})}
\] Donde \(w_{i,j}\) corresponde a peso y a la madurez \(\mu_{i,a}\) y
\(T_s\) la fracción del año correspondiente al período de desove de
sardina equivalente a \(=2/12\) correspondiente al año i y edad j,
mientras que en anchoveta \(T_s\) es \(=7/12\).

\section{Matriz de transición por
año.}\label{matriz-de-transicion-por-ano.}

\emph{Longitud media}
\[L_{a}=L_{\infty} (1-exp^{-k(a-t_{0})})\sigma_{a}\],\hspace{2cm}
\[\sigma_{j}=CV_{j}L_{j}\] \emph{Matriz transición edad y longitud} \[
\psi_{a,l}={\frac{1} {2 \sqrt{{2\pi\sigma_{a}^{2}}}}} exp^{\left(-\frac{1}{2}\sigma_{a}^{2}(L_{l}-L_{a})^2 \right) }
\]

donde: \(L_{l}\) representa la marca de clase del intervalo de longitud
\(l\) y \(L_{a}\)La es la longitud media a la edad \(a\).

\emph{Abundancia por longitud}

\[N_{t,l}=\sum_{j=1}\psi_{a,l}N_{t,a}\]

Abundancia en cruceros \[
Ns_{i,a}=q_{s}S_{a}N_{i,a}\exp^{(-(Ts)Z_{i,a})}
\] Con \(q_{s}\) coeficiente capturabilidad crucero, \(S_a\) la
selectividad, y \(Ts\) la fecha de realización que coincide con el
período de desove.

\emph{función de Selectividad} \[
sel_fish_j=\frac{1}  {1+exp^{\left({-ln19} {\frac{(a-a50)} {a95}}\right)}}
\] \emph{Mortalidad de pesca} \[
F_{t,a}=F_{t}S_{a}
\] Mortalidad total \[
Z_{a,t}=M+F_{a,t}
\]

\emph{Abundancia a la edad} Con a la edad y t el año, a={[}0,1,2,3,4{]}
con 5 edades. \[
N_{a+1,t+1}=N_{a,t}e^{(-Z_{a,t})}
\] En el caso de la última edad \(m\) y último año: \[
{N_{t+1,A}={N_{t,A-1}}exp^{-(Z_{t,A})} + {N_{t,A}}exp^{-(Z_{t,A})}}
\]

\emph{Abundancia inicial} \[
N_{a,1}=\left(R_{0}e^{-\sum_{a} aZ_{a,t=1}}\right)e^{\epsilon_a+0.5\sigma^2_R} 
\] \emph{Captura} \[
\widehat{C}_{a,t} = {\frac{F_{a,t}} {Z_{a,t}}} N_{a,t}(1-S_{a,t})
\]


\end{document}
