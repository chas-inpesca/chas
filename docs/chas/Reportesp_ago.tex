\documentclass[]{article}
\usepackage{lmodern}
\usepackage{amssymb,amsmath}
\usepackage{ifxetex,ifluatex}
\usepackage{fixltx2e} % provides \textsubscript
\ifnum 0\ifxetex 1\fi\ifluatex 1\fi=0 % if pdftex
  \usepackage[T1]{fontenc}
  \usepackage[utf8]{inputenc}
\else % if luatex or xelatex
  \ifxetex
    \usepackage{mathspec}
  \else
    \usepackage{fontspec}
  \fi
  \defaultfontfeatures{Ligatures=TeX,Scale=MatchLowercase}
\fi
% use upquote if available, for straight quotes in verbatim environments
\IfFileExists{upquote.sty}{\usepackage{upquote}}{}
% use microtype if available
\IfFileExists{microtype.sty}{%
\usepackage{microtype}
\UseMicrotypeSet[protrusion]{basicmath} % disable protrusion for tt fonts
}{}
\usepackage[margin=1in]{geometry}
\usepackage{hyperref}
\hypersetup{unicode=true,
            pdftitle={Reporte estrategias de explotacion en 2017},
            pdfauthor={Marcos Arteaga Vásquez, Claudio Gatica, Aquiles Sepúlveda},
            pdfborder={0 0 0},
            breaklinks=true}
\urlstyle{same}  % don't use monospace font for urls
\usepackage{graphicx,grffile}
\makeatletter
\def\maxwidth{\ifdim\Gin@nat@width>\linewidth\linewidth\else\Gin@nat@width\fi}
\def\maxheight{\ifdim\Gin@nat@height>\textheight\textheight\else\Gin@nat@height\fi}
\makeatother
% Scale images if necessary, so that they will not overflow the page
% margins by default, and it is still possible to overwrite the defaults
% using explicit options in \includegraphics[width, height, ...]{}
\setkeys{Gin}{width=\maxwidth,height=\maxheight,keepaspectratio}
\IfFileExists{parskip.sty}{%
\usepackage{parskip}
}{% else
\setlength{\parindent}{0pt}
\setlength{\parskip}{6pt plus 2pt minus 1pt}
}
\setlength{\emergencystretch}{3em}  % prevent overfull lines
\providecommand{\tightlist}{%
  \setlength{\itemsep}{0pt}\setlength{\parskip}{0pt}}
\setcounter{secnumdepth}{0}
% Redefines (sub)paragraphs to behave more like sections
\ifx\paragraph\undefined\else
\let\oldparagraph\paragraph
\renewcommand{\paragraph}[1]{\oldparagraph{#1}\mbox{}}
\fi
\ifx\subparagraph\undefined\else
\let\oldsubparagraph\subparagraph
\renewcommand{\subparagraph}[1]{\oldsubparagraph{#1}\mbox{}}
\fi

%%% Use protect on footnotes to avoid problems with footnotes in titles
\let\rmarkdownfootnote\footnote%
\def\footnote{\protect\rmarkdownfootnote}

%%% Change title format to be more compact
\usepackage{titling}

% Create subtitle command for use in maketitle
\newcommand{\subtitle}[1]{
  \posttitle{
    \begin{center}\large#1\end{center}
    }
}

\setlength{\droptitle}{-2em}
  \title{Reporte estrategias de explotacion en 2017}
  \pretitle{\vspace{\droptitle}\centering\huge}
  \posttitle{\par}
  \author{Marcos Arteaga Vásquez, Claudio Gatica, Aquiles Sepúlveda}
  \preauthor{\centering\large\emph}
  \postauthor{\par}
  \predate{\centering\large\emph}
  \postdate{\par}
  \date{Agosto 2017}


\begin{document}
\maketitle

\section{Resumen Ejecutivo}\label{resumen-ejecutivo}

Se desarrolla una ``Evaluación de Estrategias de explotación'' para la
pesquería de pequeños pelágicos de Chile centro Sur. El proyecto consta
de 3 etapas: a) diseño modelo operativo; b) Implementación MO; y c)
Análisis de Estrategias de explotación. La primera etapa, se inicia con
la preparación de información y consolidación de información y modelos
en actual uso. Luego, se organiza un sitio para compartir los
antecedentes con el experto internacional, Dr.~James Ianelli (NOAA,
Washington). Se coordina un taller en la ciudad de Seattle, con la
participación del experto y el grupo de investigadores de INPESCA. La
actividad se desarrolla entre los días 5-10 julio de 2017 en
dependencias de la Universidad de Washington y la NOAA. Como resultado,
se desarrolla un primer MO para la pesquería de pequeños pelágicos que
consta de una retro-alimentación con modelos de estimación de sardina
común y anchoveta. En forma complementaria se participó en un taller de
Stock Synthesis con énfasis en procesos de simulación donde se
presentaron por parte de las expertas kelliFaye Johnson -U Washington
\href{mailto:Kfjohns@uw.edu}{\nolinkurl{Kfjohns@uw.edu}} y Gwladys
Lambert - NOAA Affiliate
\href{mailto:gwladys.lambert@noaa.gov}{\nolinkurl{gwladys.lambert@noaa.gov}},
los procesos necesarios para implementar diferentes casos de modelos
operativos, haciendo uso de SS3 para la generación de datos simulados y
test de modelos de estimación.

\section{Metodologia}\label{metodologia}

La metodología para desarrollar un modelo operativo para la pesquería de
pequeños pelágicos se basa en dos enfoques. Un primer enfoque de
conocimiento de los recursos, sobre la base de documentos técnicos,
publicaciones y conocimiento experto de investigadores a cargo del
desarrollo de las evaluaciones de pequeños pelágicos. El segundo enfoque
es trabajo directo de desarrollo de modelos de dinámica, modelos
operativos y análisis de datos para alimemtar los modelos a implementar.
Por lo tanto, se requiere ir avanzando en paralelo entre conocimiento y
herramientas avanzadas de análisis.

\subsection{Coordinación y
actualización}\label{coordinacion-y-actualizacion}

La coordinación se verifica utilizando un sition de almacenamiento de
información (dropbox) y un sitio de trabajo y actualización de modelos
en github (\url{https://github.com/chas-inpesca}). En chas-inpesca, se
encuentran alojados los modelos de estimación de sardina común y
anchoveta y el \textbf{modelo operativo}, denominado
(\textbf{chas},``Chilean anchovy-sardine'' model).

Con el objetivo de mantener continuidad en el trabajo, se opta por
desarrollar un trabajo quincenal, donde los investigadores nacionales
los días miercoles consolidaran los avances en los diferentes tópicos
del MO.

\subsection{Especificación del modelo
Operativo}\label{especificacion-del-modelo-operativo}

El modelo operativo y modelos de estimación, serán específicos en la
descripción de ecuaciones de dinámica, empleando para ellos como
estandar RMarkdown para los reportes y presentación de ecuaciones y
funciones estadísticas (i.e.~distribuciones de probabilidad y máxima
verosimilitud). A modo de ejemplo se presentan las ecuaciones de:

Captura \[
C_{i,j} =N_{ij}\frac{F_{i,j}}{Z_{i,j}}[1-e^{-Z_{i,j}}]
\] Decaimiento de la abundancia \[
N_{ij}= N_{i-1j-1}e^{-Z_{ij}}
\] \# Resultados

\subsection{Taller de simulación en stock
synthesis}\label{taller-de-simulacion-en-stock-synthesis}

El primer día de trabajo en Seattle, correspondió en asistir a un WS en
las dependencias de la NOAA, a cargo de las investigadoras, kelliFaye
Johnson -U Washington
\href{mailto:Kfjohns@uw.edu}{\nolinkurl{Kfjohns@uw.edu}} y Gwladys
Lambert - NOAA Affiliate
\href{mailto:gwladys.lambert@noaa.gov}{\nolinkurl{gwladys.lambert@noaa.gov}}.
En este taller, se explicó sobre la base de un sitio en GIT:

**\url{https://github.com/ss3sim/ss3sim**}

La forma de instalación de SS3 y sus componentes, para luego proceder a
correr casos de estudio sobre un modelo base y diferentes cambios de
configuración. Los principales contenidos corresponden a:

\begin{enumerate}
\def\labelenumi{\alph{enumi})}
\item
  Instalación de SS3sim R package
\item
  Configuración de simulación ss3sim
\item
  Como trabaja SS3
\item
  Ejemplos de salidas para simulaciones ss3sim
\item
  Publicaciones utilizando ss3sim
\end{enumerate}

Los diferentes casos de ejemplo, y posibilidades dan cuenta de una
herramienta viable para testear diferentes configuraciones de modelos
con variantes de sexos, flotas y diferentes opciones en los procesos de
dinámica (i.e.~selectividad, reclutamiento, crecimiento, mortalidad
entre otros).

\subsection{Material y documentación}\label{material-y-documentacion}

Como resultado del taller de 1 día, se logro comprender la forma de
trabajo con SS3, obtener material de trabajo, ejemplos y contactos con
experiencia en el uso de SS3 como herramienta de implementación de
modelos, análisis de estrategias y procesos de simulación. Además, de
presentaciones de expertos en las posibilidades y características de uso
de SS3.

\section{Apendice 1}\label{apendice-1}

\subsection{Conclusiones.}\label{conclusiones.}

Las siguientes conclusiones claves fueron realizadas una vez revisado el
material relacionado con el workshop:

\begin{enumerate}
\def\labelenumi{\arabic{enumi}.}
\item
  Se adopta mixsan.tpl codigo como modelo operativo, incluyendo mejores
  diagnóstico para el condicionamiento (ajuste) dos especies(MCMC).
\item
  Se realiza la actualización de información para modelos de estimación
  y operativos (completado).
\item
  En relación al MPH (DEPM), se consideera que en caso deser incluido no
  tendría mucho peso.
\item
  Se solicita revisar el tratamiento del reclutamiento intra-especifíco
\item
  Se debe revisar la proporción de las capturas y modelo entre-especies.
\item
  Se reequiere continuar el trabajo con el modelo de estimación para
  testear estrategias de manejo.
\item
  Se deve modificar codigo de tal forma que el mismo codigo trabaje
  tanto para sardina como para anchoveta (entradas diferentes) y/o
  clarificar aspectos de crecimiento (existen algunas opciones para
  especificar para especificar diferentes matrices de transicion) o
  promedios.
\item
  Concluimos que no es necesario que el modelo operativo trabaje con el
  modelo y datos de ifop.
\item
  Se acuerda coordinar documentos y códigos utilizando github, dropbox
  para documentas e información y github para trabajar con windows/linux
  y mac.
\end{enumerate}

\section{Actividades realizadas}\label{actividades-realizadas}

Dado un enfoque para el Modelo operativo basado en Mixsan y ahora
demoninado (CHAS), los siguientes puntos fueron trabajados durante el
WS:

\begin{enumerate}
\def\labelenumi{\arabic{enumi}.}
\item
  Actualización set de datos hasta el 2016 (basado en año calendario).
\item
  Comparación de matrices de transición entre evaluaciones y nivel de
  ajuste.
\item
  Revisión proporción de capturas (históricas)
\item
  Evaluación posterior (combinada y separada por stocks) y exploración
  de ADMB para evaluación de posterior.
\item
  Obtener información para sardina y anchoveta para trabajar en el
  modelo de estimación (em.tpl). Este fue completado para especificar
  tiempo de desove e inclusion de peso a la edad promedio.
\item
  Testeo de modo simulación. Este fue emprendido con el set de datos
  originales y aplicado con pasos ejecutables (modelo estimación no
  totalmente revisado y actualizado).
\end{enumerate}

\section{Trabajo a realizar}\label{trabajo-a-realizar}

\begin{enumerate}
\def\labelenumi{\arabic{enumi}.}
\item
  Analizar proporción de capturas para ajuste de cuotas.
\item
  Rectificar crecimiento en relación a tiempos variantes (períodos).
\item
  Mantener datos longitud-peso organizados y parámetros .
\item
  Revisar crecimiento estacional (cruceros reclas y pelaces se
  desarrollan en diferentes periodos del año, y la pesquería. Podrían
  ser necesarias matrices de transición distintas dependiendo de cada
  fuente de información.
\end{enumerate}

\section{Notas}\label{notas}

Pasos para progreso modelo operativo

\begin{enumerate}
\def\labelenumi{\arabic{enumi})}
\tightlist
\item
  \textbf{Proceso de decisión}
\end{enumerate}

\begin{enumerate}
\def\labelenumi{\roman{enumi}.}
\item
  Condicionamiento del modelo (ajuste a datos básicos y rangos
  biológicos plausibles).
\item
  Uso de mixsan y desarrollo futuro. Modificar para lidiar don
  diferentes escalas temporales, diferencias espaciales, minimizar
  actualización con menos datos.
\item
\begin{verbatim}
Desarrollo de nuevo código.
\end{verbatim}
\item
  Generación de datos para testear procedimientos de manejo (e.g., como
  en mixsan, para testear anchoveta.tpl y sardine.tpl)
\end{enumerate}

\begin{enumerate}
\def\labelenumi{\arabic{enumi})}
\setcounter{enumi}{1}
\tightlist
\item
  \textbf{Escenarios Modelo operativo}
\end{enumerate}

\begin{enumerate}
\def\labelenumi{\alph{enumi}.}
\item
  Escenarios capturas históricas (oficial y correguido), parámetros a
  especificar.
\item
  Crecimiento (IFOP vs Longitud edad asumida por Inpesca), anual o por
  opciones de período.
\item
  MCMC sorteados para set de datos de simulación.
\end{enumerate}

Consultas

\begin{enumerate}
\def\labelenumi{\arabic{enumi})}
\item
  matriz transición talla-edad consistente con parámetros y relación
  Longitud-peso.
\item
  Disponibilidad de parámetros longitud-peso.
\item
  Implementar chas modificado con datos actualizados.
\end{enumerate}


\end{document}
