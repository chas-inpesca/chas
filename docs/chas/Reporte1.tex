\documentclass[]{article}
\usepackage{lmodern}
\usepackage{amssymb,amsmath}
\usepackage{ifxetex,ifluatex}
\usepackage{fixltx2e} % provides \textsubscript
\ifnum 0\ifxetex 1\fi\ifluatex 1\fi=0 % if pdftex
  \usepackage[T1]{fontenc}
  \usepackage[utf8]{inputenc}
\else % if luatex or xelatex
  \ifxetex
    \usepackage{mathspec}
  \else
    \usepackage{fontspec}
  \fi
  \defaultfontfeatures{Ligatures=TeX,Scale=MatchLowercase}
\fi
% use upquote if available, for straight quotes in verbatim environments
\IfFileExists{upquote.sty}{\usepackage{upquote}}{}
% use microtype if available
\IfFileExists{microtype.sty}{%
\usepackage{microtype}
\UseMicrotypeSet[protrusion]{basicmath} % disable protrusion for tt fonts
}{}
\usepackage[margin=1in]{geometry}
\usepackage{hyperref}
\hypersetup{unicode=true,
            pdftitle={Reporte 1 MSE},
            pdfauthor={Marcos Arteaga , Claudio Gatica},
            pdfborder={0 0 0},
            breaklinks=true}
\urlstyle{same}  % don't use monospace font for urls
\usepackage{graphicx,grffile}
\makeatletter
\def\maxwidth{\ifdim\Gin@nat@width>\linewidth\linewidth\else\Gin@nat@width\fi}
\def\maxheight{\ifdim\Gin@nat@height>\textheight\textheight\else\Gin@nat@height\fi}
\makeatother
% Scale images if necessary, so that they will not overflow the page
% margins by default, and it is still possible to overwrite the defaults
% using explicit options in \includegraphics[width, height, ...]{}
\setkeys{Gin}{width=\maxwidth,height=\maxheight,keepaspectratio}
\IfFileExists{parskip.sty}{%
\usepackage{parskip}
}{% else
\setlength{\parindent}{0pt}
\setlength{\parskip}{6pt plus 2pt minus 1pt}
}
\setlength{\emergencystretch}{3em}  % prevent overfull lines
\providecommand{\tightlist}{%
  \setlength{\itemsep}{0pt}\setlength{\parskip}{0pt}}
\setcounter{secnumdepth}{0}
% Redefines (sub)paragraphs to behave more like sections
\ifx\paragraph\undefined\else
\let\oldparagraph\paragraph
\renewcommand{\paragraph}[1]{\oldparagraph{#1}\mbox{}}
\fi
\ifx\subparagraph\undefined\else
\let\oldsubparagraph\subparagraph
\renewcommand{\subparagraph}[1]{\oldsubparagraph{#1}\mbox{}}
\fi

%%% Use protect on footnotes to avoid problems with footnotes in titles
\let\rmarkdownfootnote\footnote%
\def\footnote{\protect\rmarkdownfootnote}

%%% Change title format to be more compact
\usepackage{titling}

% Create subtitle command for use in maketitle
\newcommand{\subtitle}[1]{
  \posttitle{
    \begin{center}\large#1\end{center}
    }
}

\setlength{\droptitle}{-2em}
  \title{Reporte 1 MSE}
  \pretitle{\vspace{\droptitle}\centering\huge}
  \posttitle{\par}
  \author{Marcos Arteaga , Claudio Gatica}
  \preauthor{\centering\large\emph}
  \postauthor{\par}
  \predate{\centering\large\emph}
  \postdate{\par}
  \date{17 Agosto 2017}


\begin{document}
\maketitle

\section{Resumen}\label{resumen}

Se desarrolla una reunión de trabajo, con el objeto de priorizar tareas
identificadas en WS MSE. Sobre la base del reporte del WS en Seattle. Se
revisa la disposición de información, organización de modelos de
estimación y operativos. En relación a la información, se genera un
listado de datos a actualizar para la evaluación, análisis necesarios y
nuevos datos a requerir.

\section{Datos}\label{datos}

\begin{enumerate}
\def\labelenumi{\arabic{enumi}.}
\item
  Actualización set de datos hasta el 2016 (basado en año calendario).
\item
  Analizar proporción de capturas para ajuste de cuotas.
\item
  Rectificar crecimiento en relación a tiempos variantes (períodos).
\item
  Mantener datos longitud-peso organizados y parámetros .
\item
  Revisar crecimiento estacional (cruceros reclas y pelaces se
  desarrollan en diferentes periodos del año, y la pesquería. Podrían
  ser necesarias matrices de transición distintas dependiendo de cada
  fuente de información.
\end{enumerate}

\section{Análisis}\label{analisis}

\begin{enumerate}
\def\labelenumi{\arabic{enumi})}
\item
  Se revisara el proceso de reclutamiento por especie, caracterizando
  tendencias, magnitudes y representacion a partir del Modelo de
  estimación. (Artega, Gatica)
\item
  Se debe revisar la proporción de las capturas y modelo entre-especies.
\item
  Escenarios capturas históricas (oficial y correguido), parámetros a
  especificar.
\item
  matriz transición talla-edad consistente con parámetros y relación
  Longitud-peso.
\end{enumerate}

\section{Codificacion}\label{codificacion}

\begin{enumerate}
\def\labelenumi{\arabic{enumi}.}
\tightlist
\item
  Se deve modificar codigo de tal forma que el mismo codigo trabaje
  tanto para sardina como para anchoveta (entradas diferentes) y/o
  clarificar aspectos de crecimiento (existen algunas opciones para
  especificar para especificar diferentes matrices de transicion) o
  promedios.
\end{enumerate}

\begin{enumerate}
\def\labelenumi{\arabic{enumi})}
\setcounter{enumi}{1}
\tightlist
\item
  Disponibilidad de parámetros longitud-peso.
\end{enumerate}


\end{document}
